\documentclass[a4paper]{article}\addtolength{\hoffset}
{-2.25cm}
\addtolength{\textwidth}
{5cm}
\addtolength{\voffset}
{-3.25cm}
\addtolength{\textheight}
{5.5cm}
\setlength{\parskip}{0pt}
\setlength{\parindent}{0in}

\usepackage[utf8]{inputenc}
\usepackage{microtype}
\usepackage[english]{babel}
\usepackage{fancyhdr}
\usepackage{amsmath, amssymb}
\usepackage{graphicx}

\graphicspath{{./images/}}

\begin{document}

\fancyhead[c]{}
\hrule \medskip
\begin{minipage}{0.295\textwidth}
\raggedright
Rishabh Indoria
\end{minipage}
\begin{minipage}{0.4\textwidth}
\centering
\LARGE
Machine Learning Techniques
\end{minipage}\
\begin{minipage}{0.295\textwidth}
\raggedleft
\today \hfill \\
\end{minipage}
\medskip \hrule
\bigskip

\section{Week 1}
\begin{enumerate}
    \item{Paradigms of Machine Learning}
    \begin{itemize}
        \item Broad Paradigms: Supervised Learning, Unsupervised Learning, Sequential Learning
        \item Foundations: Linear Algebra for Structure, Probability for Uncertainty, Optimization for Decision
    \end{itemize}
    \item{Representation Learning}
    \begin{itemize}
        \item Part of Unsupervised Learning
        \item \textbf{Compression}: Act of finding patterns in data.\\
        \textbf{Representation}: Some sort of relation within the data point\\
        \textbf{Coefficients}: Part of the data point, which when written as the representation, we could get the original data point back.
        \item Choose Representation and Coefficient such that reconstruction error is minimized.
        \item $n$ represents the number of data points and $d$ represents the number of features.
        \item Projection of $\textbf{x}$ onto line $\textbf{w}$ is $\frac{\textbf{x.w}}{||\textbf{w}||}\frac{\textbf{w}}{||\textbf{w}||}$
        \item Optimal value of $\textbf{w}$ is the eigenvector corresponding to the maximum eigenvalue of the covariance matrix. If there are $d$ features, then the Covariance matrix will be a $d\times d$ matrix. The Covariance matrix is $\frac{1}{n}\sum_1^nx_ix_i^T$.
        \item We want to minimize $\textbf{w}C\textbf{w}.T$.
        \item All the residuals $\textbf{x} - (\textbf{x.w})\textbf{w}$ are perpendicular to the line $\textbf{w}$
        \item \textbf{Centred Data}: the Mean value of the dataset is 0. To centre a dataset, you simply subtract the dataset by the mean of the dataset.
    \end{itemize}
    \item{Principal Component Analysis}
    \begin{itemize}
        \item $\textbf{w}_1$ is the line which minimizes the reconstruction error of a set of points, and $\textbf{w}_2$ is the line which minimizes the reconstruction error of the residues. These lines are orthogonal to each other.
        \item We find the best fit line $\textbf{w}_1$, then we find residuals, then using the residuals we find the best fit again $\textbf{w}_2$. We keep doing this for $d$ iterations. After which, residues will definitely become 0. The final residue will be $\textbf{x} - (\textbf{x.w}_1)\textbf{w}_1 - (\textbf{x.w}_2)\textbf{w}_2 - ...$
        \item The vectors $\textbf{w}_1,\textbf{w}_2,...,\textbf{w}_d$ are orthogonal and span or are the basis of $\mathbb{R}^d$.
        \item The best line one can obtain at $k$th round of the algorithm is the eigenvector corresponding to the $k$th maximum eigenvalue of the Covariance matrix.
    \end{itemize}
\end{enumerate}
\section{Week 2}
\begin{enumerate}
    \item Concerns in PCA
    \begin{itemize}
        \item Time Complexity: Finding the eigenvalues and eigenvectors, takes about O($d^3$). Issue when $d$ is large.
        \item It tries to find linear combinations as such, non-linear relationships do not fit well.
    \end{itemize}
    \item Time complexity Issue
    \begin{itemize}
        \item Large $d$ [$d$ $>>$ $n$]; $X$ is $n\times d$
        \item Let $w_k$ be the eigenvector corresponding to the $k$th largest eigenvalue of C($\lambda_k$).\\
        C$w_k$ = $\lambda_kw_k$
        \item $w_k$ = $X\alpha_k$, $\alpha_kX^TX\alpha_k$ = 1
        \item Non zero eigenvalues of $XX^T$ and $X^TX$ are exactly the same.
        \item $\beta_k$ are eigenvectors corresponding to $n\lambda_k$ of $X^TX$, converting them according to constraint $\alpha_k$ = $\frac{\beta_k}{\sqrt{n\lambda_k}}$
        \item Compute K = $X^TX$, Compute eigen decomposition, convert eigenvectors according to constraint, then finally $w_k$ = $X\alpha_k$
    \end{itemize}
    \item Feature Transformation: Increase the dimensions, such that non-linear relationships are captured, then apply PCA.
    \item Kernel function
    \begin{itemize}
        \item To convert from quadratic to linear, we map the features to $\phi(x)$ = [$1$ $f_1^2$ $f_2^2$ $f_1f_2$ $f_1$ $f_2$]
        \item Any function $k$: $\mathbb{R}^d\times\mathbb{R}^d\rightarrow \mathbb{R}$ which is a valid map is called a kernel function.
        \item A function k is a valid kernel function if there exists a function $\phi:$ $\mathbb{R}^d\rightarrow \mathbb{R}^D$ such that $k(x_1,x_2)$ = $\phi(x_1^T)\phi(x_2)$ for all $x_1,x_2\in \mathbb{R}^d$
        \item Kernel is symmetric and positive semi definite. All eigenvalues of $k$ are non-negative.
        \item Polynomial Kernel: $k(x,x')$ = $(x^Tx' + 1)^2$
        \item Radial Basis function kernel or Gaussian Kernel: $k(x,x')$ = $e^{\frac{-||x-x'||^2}{2\sigma^2}}$
    \end{itemize}
    \item Kernel PCA
    \begin{itemize}
        \item Compute Kernel matrix using a kernel function
        \item Center the Kernel using the formula: $K^C$ = $K-1_nK-K1_n+1_nK1_n$ = $(I-1_n)K(I-1_n)$, where $1_n$ is a matrix with all elements $\frac{1}{n}$
        \item Compute eigenvalues $\{n\lambda_1, n\lambda_2, ...\}$ and eigenvectors $\{\beta_1, \beta_2, ...\}$ of K and normalize the eigenvectors, $\alpha_u$ = $\frac{\beta_u}{\sqrt{n\lambda_u}}$.
        \item Compute the transformed data points, $\phi(x_i)^Tw$ = [$\sum \alpha_{1j}K^C_{ij}, \sum \alpha_{1j}K^C_{ij}, ...$]
        \item We cannot 'recompute' the eigenvectors of the covariance matrix as they would require computing, $\phi$ which would defeat the whole purpose.
        \item But we can still compute a "compressed" representation.
    \end{itemize}
\end{enumerate}
\section{Week 3}
\begin{enumerate}
    \item Intro to Clustering
    \begin{itemize}
        \item Goal: Partition the given data into k different clusters.
        \item Data points: $\{x_1, x_2, ...\}$
        \item Cluster Indicator: $\{z_1, z_2, ...\}$
        \item Performance Matrix: $F(z_1, ..., z_n)$ = $\sum_{i=1}^n||x_i - \mu_{z_i}||^2_2$, where $\mu_{z_i}$ is the mean/average of $z_i$ cluster.
        \item Goal is to minimize Performance matrix
    \end{itemize}
    \item K-means Clustering
    \begin{itemize}
        \item Also known as Lloyd's Algorithm
        \item Step 1, Initialization: We define some assignment to clusters $z_1^0, z_2^0, ..., z_n^0$
        \item Then until convergence we
        \item Step 2, Compute means: $\mu_k^t$ = $\frac{\sum x_i1(z_i^t=k)}{\sum 1(z_i^t=k)}$
        \item Step 3, Reassignment: $z_i^{t+1}$ = arg min$_k$ $||x_i - \mu_k^t||^2_2$
        \item FACT: LLOYD'S ALGORITHM CONVERGES, but converged solution may not be "optimal". But produces "reasonable" cluster in practice.
    \end{itemize}
    \item Convergence of K-means
    \begin{itemize}
        \item FACT 1: Let $x_1,x_2,...,x_l$, $v^*$ = arg min$_v$ $\sum ||x_i - v||^2$. $v^*$ = $\frac{\sum x_i}{l}$
        \item In every iteration, the objective function strictly reduces, which implies that no partition repeats.
        \item There are only "FINITE" number of partitions as such algorithm must converge.
    \end{itemize}
    \item Nature of Clusters
    \begin{itemize}
        \item For a cluster with mean $\mu_1$, all $x$ assigned to it will satisfy $x^T(\mu_n - \mu_1) \leq \frac{||\mu_n||^2 - ||\mu_1||^2}{2}$, for all $n\neq 1$
        \item VORONOI region: intersection of half spaces. Cluster regions are voronoi regions.
        \item K-means can not efficiently cluster data points that are not linearly separable. Kernel K-means is used to cluster data points that are not linearly separable. Spectral Clustering can also be used.
    \end{itemize}
    \item Initialization of centroids
    \begin{itemize}
        \item Pick K-means uniformly at random from the dataset.
        \item Means should be far apart.
        \item K-means++: Choose first mean $\mu^0_1$ uniformly at random from the dataset. For $l=2,3,...,k$ choose $\mu_l^0$ probabilistically proportional to score.
        \item Score $S(x)$ = min$_j$ $||x-\mu_j^0||^2$
    \end{itemize}
    \item Choice of K
    \begin{itemize}
        \item We want K to be not as small and not as large. Penalize large values of K.
        \item Value of K is where Objective function + Penalty function is the smallest.
        \item Akaike Information Criterion: $2K-2\log(L(\theta^*))$
        \item Bayesian Information Criterion: $K\log(n)-2\log(L(\theta^*))$
    \end{itemize}
\end{enumerate}
\section{Week 4}
\begin{enumerate}
    \item Introduction to Estimation
    \begin{itemize}
        \item Estimation: There is some probabilistic mechanism that generates the data. About which we don't know "something".
        \item Goal: Observe data and "Assume" a probabilistic model that generates the data.
        \item Assumption: Observations are \textbf{Independent} and \textbf{Identically Distributed}
    \end{itemize}
    \item Maximum Likelihood Estimation
    \begin{itemize}
        \item Fisher's Principle of Maximum Likelihood: Write the likelihood function \\
        $L(p, \{x_1,x_2,...x_n\})$ = $P(x_1,x_2,...,x_n,p)$\\
        = $P(x_1,p).P(x_2,p)...P(x_n,p)$ = $\Pi_{i=1}^np^{x_i}(1-p)^{1-x_i}$ [Independence]
        \item Estimator: $\hat{p}_{ML}$ = arg max$_p\Pi_{i=1}^np^{x_i}(1-p)^{1-x_i}$\\
        We take logarithm to simplify\\
        = arg max$_p\sum_{i=1}^nx_i\log(p)+(1-x_i)\log(1-p)$ [log is monotonically increasing]\\
        Taking Derivative and setting to 0 we get,\\
        $\hat{p}_{ML}$ = $\frac{\sum_{i=1}^nx_i}{n}$
        \item Fisher's Proposal: $L(\mu, \sigma^2, \{x_1,x_2,...x_n\})$ = $f_{x_1,x_2,...x_n}(x_1,x_2,...x_n,\mu,\sigma^2)$ = $\Pi f_{x_i}(x_i,\mu,\sigma^2)$\\
        This is done because for continuous functions, Probability only exists for intervals and probability at any particular point would be 0.
    \end{itemize}
    \item Bayesian Estimation
    \begin{itemize}
        \item Goal: Incorporate "Hunch" about parameters into the estimation procedure.
        \item Approach: Think of the parameter to estimate as a "random" variable.
        \item Hunch: codified using a probabilistic distribution over $\theta$
        \item After looking at data, move to\\
        Updated Hunch: Codified using another probabilistic distribution.
        \item $P(\theta | \{x_1,x_2,...x_n\})$ = $\frac{P(\{x_1,x_2,...x_n\} | \theta)P(\theta)}{P(\{x_1,x_2,...x_n\})}$
        \item BETA PRIOR: $f(p, \alpha, \beta)$ = $\frac{p^{\alpha-1}(1-p)^{\beta-1}}{z}$
        \item BETA POSTERIOR($\alpha+n_h,\beta+n_t$)
        \item One possible guess = $\frac{\alpha+n_h}{\alpha+\beta+n}$
    \end{itemize}
    \item Gaussian Mixture Models
    \begin{itemize}
        \item STEP 1: Pick which mixture a data point comes from.
        \item STEP 2: Generate data point from that mixture.
        \item Generate a mixture component among $\{1,...,k\}$, $z_i\epsilon\{1,...,k\}$\\
        $P(z_i=l)$ = $\pi_l$
        \item Generate $x_i$ = $N(\mu_{z_i}, \sigma^2_{z_i})$
        \item Observed: $\{x_1,x_2,...,x_n\}$\\
        Unobserved: $\{z_1,z_2,...,z_n\}$\\
        Parameters: $\pi$ = [$\pi_1,\pi_2,...,\pi_k$], and for each $k$ ($\mu_k, \sigma_k^2$)
    \end{itemize}
    \item Likelihood of GMM
    \begin{itemize}
        \item $L(Parameters,\{x_1,x_2,...,x_n\})$ = $\Pi_{i=1}^n f(x_i; Parameters)$\\
        = $\Pi_{i=1}^n [\sum_{k=1}^k \pi_k f(x_i;\mu_k, \sigma^2_k)]$\\
        $L(Parameters)$ = $\Pi_{i=1}^n [\sum_{k=1}^k\pi_k\frac{e^{\frac{-(x_i - \mu_k)^2}{2\sigma_k^2}}}{\sqrt{2\pi}\sigma_k}]$
        \item $\log(L(Parameters))$ = $\sum_{i=1}^n\log(\sum_{k=1}^k\pi_k\frac{e^{\frac{-(x_i - \mu_k)^2}{2\sigma_k^2}}}{\sqrt{2\pi}\sigma_k})$
    \end{itemize}
    \item Convex Functions and Jensen's inequality
    \begin{itemize}
        \item Convex Functions: $f(\frac{a+b}{2}) \leq \frac{f(a) + f(b)}{2}$ $\forall a,b$\\
        \item Jensen's inequality: $f(\lambda_1a_1 + ... + \lambda_ka_k)\leq \lambda_1f(a_1) + ... + \lambda_kf(a_k)$\\
        $f(\sum \lambda_ka_k)\leq\sum\lambda_kf(a_k)$, $\sum \lambda_k = 1$
        \item Concave Functions: $f(\frac{a+b}{2}) \geq \frac{f(a) + f(b)}{2}$ $\forall a,b$
        \item Jensen's inequality: $f(\lambda_1a_1 + ... + \lambda_ka_k)\geq \lambda_1f(a_1) + ... + \lambda_kf(a_k)$\\
        $f(\sum \lambda_ka_k)\geq\sum\lambda_kf(a_k)$, $\sum \lambda_k = 1$
        \item Linear Functions are both Concave and Convex.
        \item Log is a concave function
    \end{itemize}
    \item Estimating the parameters
    \begin{itemize}
        \item $\log(L(Parameters))$ = $\sum_{i=1}^n\log(\sum_{k=1}^k \lambda_k^i (\pi_k\frac{e^{\frac{-(x_i - \mu_k)^2}{2\sigma_k^2}}}{\lambda_k^i\sqrt{2\pi}\sigma_k}))$\\
        Apply Jensen's inequality, Fix $\lambda$ and then take derivative and set to 0
        \item $\hat{\mu_k}$ = $\frac{\sum_{i=1}^n\lambda_k^ix_i}{\sum_{i=1}^n\lambda_k^i}$, $\hat{\sigma_k^2}$ = $\frac{\sum_{i=1}^n\lambda_k^i(x_i - \hat{\mu_k})^2}{\sum_{i=1}^n\lambda_k^i}$, $\hat{\pi_k}$ = $\frac{\sum_{i=1}^n\lambda_k^i}{n}$
        \item Fixing all parameters and maximizing with respect to $\lambda$\\
        $\hat{\lambda_k^i}$ = $\frac{\frac{1}{\sqrt{2\pi}\sigma_k}e^{\frac{-(x_i - \mu_k)^2}{2\sigma_k^2}} . \pi_k}{\sum_{l=1}^k\frac{1}{\sqrt{2\pi}\sigma_l}e^{\frac{-(x_i - \mu_l)^2}{2\sigma_l^2}} . \pi_l}$ = $\frac{P(x_i|z_i=k)P(k)}{P(x_i)}$\\
        $\forall i$, $\sum_{k=1}^K\lambda^i_k$ = $1$
    \end{itemize}
    \item Expectation Maximization Algorithm
    \begin{itemize}
        \item Initialize Parameters
        \item Then until convergence\\
        Expectation Step: Calculate $\lambda^{t+1}$ using $Parameters^t$\\
        Maximization Step: Calculate $Parameters^{t+1}$ using $\lambda^{t+1}$
    \end{itemize}
\end{enumerate}
\section{Week 5}
\begin{enumerate}
    \item Supervised Learning
    \begin{itemize}
        \item Input is features/attributes $\{x_1,...x_n\}$ and labels $\{y_1,...,y_n\}$.
        \item If labels only take two values, then the problem is called binary classification problem. If labels take $n$ values, then the problem is called multi class classification. If labels take any value in real number, then the problem is called a regression problem.
    \end{itemize}
    \item Linear Regression
    \begin{itemize}
        \item Goal is to learn a function $f$ which maps a given feature to its correct label.
        \item Error($f$) = $\sum(f(x_i) - y_i)^2$
        \item For linear regression we take $f(x)$ = $w^Tx$, and minimize Error($f$).
        \item $\begin{bmatrix}---x_1---\\---x_2---\\---x_n---\end{bmatrix}\begin{bmatrix}w_1\\w_2\\w_n\end{bmatrix}$ = $\begin{bmatrix}y_1\\y_2\\y_n\end{bmatrix}$, we want to minimize $(x^Tw - y)^2$ or $(x^Tw-y)^T(x^Tw-y)$.
        \item Simply take gradient and set to $0$, we get $(xx^T)w^*$ = $xy$.
        \item Since $w^*$ is a solution of an unconstrained optimization problem, we can apply gradient descent \\
        $w^{t+1}$ = $w^t - \eta^t \triangledown f(w^t)$, where $\eta$ is the step size.\\
        $w^{t+1}$ = $w^t - \eta^t 2(xx^Tw^t - xy)$
        \item In case number of data points is large, we use\\
        \textbf{Stochastic Gradient Descent}: For $t$ iterations, sample a bunch($k$) of data points uniformly at random from the set of all points. Pretend this sample is the entire dataset and take a gradient step w.r.t it. After all rounds, we use $w^T_{SGD}$ = $\frac{1}{T}\sum w^t$.
        \item \textbf{Kernel Regression} is similar to kernel PCA, we take $w^*$ = $x\alpha^*$ and $K$ = $x^Tx$. $\alpha^*$ = $K^{-1}y$.\\
        To make a prediction $w^*\phi(x_{test})$ = $\sum_{i=1}^n\alpha_i^*K(x_i,x_{test})$
        \item \textbf{Probabilistic Regression}, assume labels are generated as $w^tx + \epsilon$, where $\epsilon$ is noise generated from a Gaussian distribution. Using Maximum Likelihood function we simply arrive at\\
        $\hat{w_{ML}}$ = min $\sum(w^Tx - y)^2$.
    \end{itemize}
\end{enumerate}
\section{Week 6}
\begin{enumerate}
    \item Goodness of Maximum Likelihood Estimator
    \begin{itemize}
        \item To understand how good $\hat{w}_{ML}$ is in estimating $w$, $E[||\hat{w}_{ML} - w||^2]$ = $\sigma^2 trace((xx^T)^{-1})$
        \item $trace((xx^T)^{-1})$ = $\sum\frac{1}{\lambda_i}$, where $\lambda_i$ are the eigenvalues of $xx^T$
        \item $\hat{w}_{new}$ = $(xx^T+\lambda I)^{-1}xy$, $trace((xx^T+\lambda I)^{-1})$ = $\sum\frac{1}{\lambda_i + \lambda}$
        \item \textbf{Existence Theorem}, There exists some $\lambda$ such that $\hat{w}_{new}$ has lesser mean squared error than $\hat{w}_{ML}$. We find this $\lambda$ by cross validation.
        \item Split the training set into train set and validation set. Train on the train set and check for error on validation set. Pick $\lambda$ that gives the least error.
        \item \textbf{K-fold Cross Validation}, Split dataset into K folds, train on K-1 folds and validate on the last fold. Pick $\lambda$ that gives the least average error.
        \item \textbf{Leave One Out Cross Validation}, train on $n-1$ data points and validate on the last point.
    \end{itemize}
    \item Bayesian Modelling for Linear Regression
    \begin{itemize}
        \item Need a Prior on $w$, a choice of prior $N(0, \gamma^2I)$
        \item Posterior is proportional to $P(dataset|w)P(w)$ = $(\prod e^{\frac{-(y_i - w^Tx_i)^2}{2}})e^{\frac{-||w||^2}{2\gamma^2}}$
        \item $\hat{w}_{MAP}$ = min $\frac{1}{2}\sum(y_i-w^Tx_i)^2 + \frac{1}{2\gamma^2}||w||^2$, taking gradient and setting to 0 we get\\
        $\hat{w}_{MAP}$ = $(xx^T + \frac{1}{\gamma^2}I)^{-1}xy$
    \end{itemize}
    \item Ridge Regression
    \begin{itemize}
        \item $\hat{w}_R$ = arg min $\sum(w^Tx_i - y_i)^2 + \lambda||w||^2$, where the added term is called regularization term.
        \item Ridge pushes weight values towards 0 but does not necessarily make it 0.
    \end{itemize}
    \item Lasso Regression
    \begin{itemize}
        \item An alternate way is to regularize would be using L1 norm(Summation of absolute values instead of squared values).
        \item Much more likely to make some weight values 0. But it does not have a closed form solution.
        \item Sub gradients methods are usually used to solve LASSO.
    \end{itemize}
\end{enumerate}
\section{Week 7}
\begin{enumerate}
    \item Binary Classification
    \begin{itemize}
        \item Labels belong to the set $\{0,1\}$ or the set $\{-1,1\}$
        \item Loss($h$) = $\frac{1}{n}\sum 1(h(x_i) \neq y_i)$
        \item $h(x)$ = $sign(w^Tx)$
    \end{itemize}
    \item K Nearest Neighbours
    \begin{itemize}
        \item Given a test point $x_{test}$, find the closest point $x'$ to $x_{test}$ in the training set. Predict $y_{test}$ = $y'$.
        \item Can get affected by outliers, Ask more neighbours and predict the majority.
        \item Problems: Choosing a distance function, Prediction is computationally expensive, No model is learnt.
    \end{itemize}
    \item Decision Trees
    \begin{itemize}
        \item A question is a (feature, value) pair. Is feature $\leq$ value ?
        \item Need a measure of "Impurity" for a set of labels to determine how good a question is.
        \item Entropy function = $-(p\log(p) + (1-p)\log(1-p))$, where p is the fraction of 1's, and $\log(0)$ is considered 0.
        \item Information Gain(feature, value) = Entropy(D) - [$\gamma$Entropy(D$_{yes}$) + (1-$\gamma$)Entropy(D$_{no}$)],\\
        where $\gamma$ = $\frac{|\text{D}_yes|}{|\text{D}|}$
        \item Discretize each feature in [min, max] range. Pick the question that has the largest Information Gain. Repeat the procedure for D$_{yes}$, D${no}$.
        \item Can stop growing a tree if a node becomes "Sufficiently" Pure. Depth of the tree is a hyperparameter. There are alternate measures for "goodness" of a question.
        \item Gini Index function is another popular function to measure impurity.
    \end{itemize}
    \item Types of Modelling
    \begin{itemize}
        \item Generative Model: $P(x, y)$
        \item Discriminative Model: $P(y|x)$
    \end{itemize}
\end{enumerate}
\section{Week 8}
\begin{enumerate}
    \item Generative Model based Algorithm
    \begin{itemize}
        \item Data: $\{(x_1,y_1), ..., (x_n,y_n)\}$, where $x\epsilon\{0,1\}^d$ and $y\epsilon\{0,1\}$.
        \item Step 1: Decide the labels by tossing a coin with $P(y_i = 1)$ = $p$
        \item Step 2: Determine the features using the labels obtained in Step 1 through the conditional probability $P(x_i|y_i)$.
        \item The parameters in generative modelling are defined as $\hat{p}$ to decide the label, $2^d-1$ parameters for $P(x|y=1)$ and $2^d-1$ parameters for $P(x|y=0)$. Where $d$ is the number of features.
        \item Too many parameters, could lead to overfitting and the model may not be practically viable.
    \end{itemize}
    \item Alternate Generative Model
    \begin{itemize}
        \item Class conditional independence: This assumption states that the features of an object are conditionally independent given its class label.
        \item Step 1 remains the same.
        \item Step 2: Determine the features for $x$ given $y$ using the following conditional probability,\\
        $P(x=[f_1,f_2,...f_n]|y)$ = $\prod(p_i^{y_i})^{f_i}(1-p_i^{y_i})^{f_i}$.
        \item The parameters in generative modelling are defined as $\hat{p}$ to decide the label, $d$ parameters for $P(x|y=1)$ and $d$ parameters for $P(x|y=0)$. Where $d$ is the number of features.
        \item Parameters are estimated using Maximum Likelihood Estimator.
    \end{itemize}
    \item Naive Bayes Algorithm
    \begin{itemize}
        \item The model is given by: $P(x=[f_1,f_2,...f_n]|y)$ = $\prod(p_i^{y_i})^{f_i}(1-p_i^{y_i})^{f_i}$.
        \item The parameters estimated are $p$, $\{p_1^0,p_2^0,...,p_d^0\}$, and $\{p_1^1,p_2^1,...,p_d^1\}$.
        \item The estimates are $\hat{p}$ = $\frac{1}{n}\sum y$, and $\hat{p}^y_j$ = $\frac{\sum_{i=1}^n 1(f_j^i=1,y_i=y)}{\sum_{i=1}^n 1(y_i=y)}$
        \item Given $x^{test}\epsilon\{0,1\}^d$, the prediction of $\hat{y}^{test}$ is done using the inequality $P(\hat{y}^{test}=1|x^{test})\geq P(\hat{y}^{test}=0|x^{test})$.
        \item Can express $P(\hat{y}^{test}=t|x^{test})$ = $\frac{P(x^{test}|\hat{y}^{test}=t)P(\hat{y}^{test}=t)}{P({x}^{test})}$, since we are only comparing, there is no need to calculate $x^{test}$.
        \item One prominent issue with Naive Bayes is that if a feature is not observed in the training set, but present in the testing set, the prediction probabilities for both classes become zero.\\
        Laplace smoothing: A popular remedy for this issue is to introduce two “pseudo” data points with labels 1 and 0, respectively, into the dataset, where all their features are set to 1.
        \item The decision function of Naive Bayes is linear, and the boundary is given by\\
        $\{x = P(y=0|x) = P(y=1| x)\}$
    \end{itemize}
    \item Gaussian Naive Bayes
    \begin{itemize}
        \item Assumes that features in the dataset follow a normal distribution and computes the likelihood of a class for a given set of feature values by estimating the mean and variance of the feature values within each class.
        \item $P(x|y=0)$ = $N(\mu_0,\Sigma)$ and $P(x|y=1)$ = $N(\mu_1,\Sigma)$
        \item $\hat{\mu}_t$ = $\frac{\sum1(y_i=t)x_i}{\sum1(y_i=t)}$, $\hat{\Sigma}$ = $\frac{1}{n}\sum(x_i - \hat{\mu}_{y_i})(x_i - \hat{\mu}_{y_i})^T$.
        \item If the covariance matrices are equal then the decision boundary is linear, if they are unequal then the decision boundary is quadratic.
        \item For unequal covariance matrix $\hat{\Sigma}_t$ = $\frac{\sum(1(y_i=t)x_i - \hat{\mu}_{t})(1(y_i=t)x_i - \hat{\mu}_{t})^T}{\sum1(y_i=t)}$.
    \end{itemize}
\end{enumerate}
\section{Week 9}
\begin{enumerate}
    \item Perceptron Learning Algorithm
    \begin{itemize}
        \item Widely employed for binary classification, focuses on modelling the boundary between each class.
        \item Objective function, $\sum 1(h(x_i)\neq y_i)$
        \item Until convergence, select a pair $(x_i,y_i)$, if, $sign(w^Tx_i)\neq y_i$ then update the weight vector\\
        $w^{t+1}$ = $w^t + x_iy_i$.
        \item $l^2\gamma^2\leq ||w_{l+1}||^2\leq lR^2$
        \item Uber bound on number of mistakes\\
        $\#mistakes \leq \frac{R^2}{\gamma^2}$
    \end{itemize}
    \item Logistic Regression
    \begin{itemize}
        \item Objective is to estimate the probability that the dependant variable belongs to one of two possible values.
        \item Let $z=w^Tx_i$, we define the $P(y=1|x)$ = $g(z)$ = $\frac{1}{1 + e^{-z}}$, where the function $g(z)$ is called the sigmoid function.
        \item Objective is to maximize the log likelihood or minimize the negative log likelihood.\\
        $y_i\log(g(z)) + (1-y_i)\log(1-g(z))$
        \item For gradient descent, we get the gradient as $x_i(y_i - g(z))$
    \end{itemize}
\end{enumerate}
\section{Week 10}
\begin{enumerate}
    \item Support Vector Machines
    \begin{itemize}
        \item category of supervised learning algorithms designed for classification and regression analysis. SVMs aim to identify the optimal hyperplane that maximizes the margin between data points from different classes.
        \item Hard Margin SVMs: applicable only when the dataset is linearly separable\\
        Direct or Kernelized Calculation of \textit{Q}: Compute the matrix $Q$ = $X^TX$ directly or using a kernel, based on the dataset.\\
        Gradient Descent: Employ the gradient of the dual formula, $\alpha^T1 - \frac{1}{2}\alpha^TY^TQY\alpha$, in a gradient descent algorithm to iteratively find a satisfactory set of Lagrange multipliers $\alpha$.\\
        label($x_{test}$) = sign($w^Tx_{test}$) = sign($\sum\alpha_iy_i(x_i^Tx_{test})$)\\
        label($x_{test}$) = sign($w^T\phi(x_{test})$) = sign($\sum\alpha_iy_ik(x_i^Tx_{test})$)
        \item Soft Margin SVMs: extends the standard SVM algorithm to accommodate some misclassifications in the training data. This extension is particularly useful when dealing with non-linearly separable data. It introduces a regularization parameter ($C$) to control the balance between maximizing the margin and allowing for misclassifications.\\
        min $\frac{1}{2}||w||^2_2 + C\sum\epsilon_i$ such that $(w^Tx_i)y_i + \epsilon_i \geq 1$, $\epsilon_i \geq 0$
    \end{itemize}
\end{enumerate}
\section{Week 11}
\begin{enumerate}
    \item Bagging
    \begin{itemize}
        \item Simply Distribute the dataset into m smaller datasets, them make m different models. For prediction, predict using each of the models, average out the prediction, then use the same function on the averaged prediction.
        \item Can also use feature bagging.
    \end{itemize}
    \item Boosting
    \begin{itemize}
        \item Input: S = $\{(x_1,y_1),...,(x_n,y_n)\}$
        \item Initialize $D_0(i) = \frac{1}{n}$
        \item For t = 1 to T\\
        $h_t$ = Input S to a weak Learner\\
        $\Tilde{D}_{t+1}(i)$ = $D_t(i)e^{\alpha_t}$ if $h_t(x_i)\neq y_i$ else $D_t(i)e^{-\alpha_t}$\\
        $D_{t+1}(i) = \frac{\Tilde{D}_{t+1}(i)}{\sum\Tilde{D}_{t+1}(i)}$
        \item $\alpha_t$ = $\log(\sqrt{\frac{1 - error(h_t)}{error(h_t)}})$
        \item $h^*(x)$ = sign($\sum\alpha_th_t(x)$)
    \end{itemize}
\end{enumerate}
\section{Week 12}
\begin{enumerate}
    \item Activation Functions
    \begin{itemize}
        \item Sigmoid function: $\frac{1}{1+e^{-z}}$
        \item Rectified Linear Unit: max(0,z)
    \end{itemize}
\end{enumerate}

\end{document}
